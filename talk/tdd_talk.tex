% -*- latex -*-
\documentclass[landscape]{slides}

%\usepackage[landscape,pdftex]{geometry}  % this creates landscape mode in 
                                         % pdflatex, It is mutually exclusive
                                         % with \documentclass[landscape] 

\usepackage{url}                         % web addresses
\usepackage{graphicx}
\usepackage{color}                       % color glyphs
% define some colors
\definecolor{mediumGray}{gray}{0.5}
\definecolor{lightGray}{gray}{0.7}
\renewcommand{\title}[1]{{\large\bfseries #1}}

\usepackage{array}                       % extra row separation
\setlength{\extrarowheight}{10pt}        % replace bullets with vspace


\newenvironment{itemiz}%
  {\begin{list}{}{\raggedright
      \setlength{\itemsep}{2pt}%
      \setlength{\parskip}{4pt}\setlength{\parsep}{2pt}}}%
  {\end{list}}%

\setlength{\voffset}{-0.5in}
\setlength{\textheight}{7.0in}
\setlength{\hoffset}{-0.5in}
\setlength{\textwidth}{9.0in}

\begin{document}\raggedright

  \begin{titlepage}
    \thispagestyle{empty}
     {\bfseries\Large Ongoing Research Into Numerical Simulation 
       of Fluid Flows Utilizing Software Development Practices}\\
     \vfill
	 {\large FUN3D Software Development Team} \\
	 {\itshape NASA, Hampton, Virginia}
	 \vfill
	     {ACDL Seminar\\
	       by MikePark@MIT.Edu Mike.Park@NASA.Gov\\
	       24 September 2004}\\
  \end{titlepage}

  \begin{slide}
    \title{FUN3D}

    Originated by Kyle Anderson, Eric Nielsen, and others

    Extended by High Energy Flow Solver Synthesis (HEFSS) effort

    An element of the Fast Adaptive AeroSpace Tools (FAAST) project

    Detailed in \emph{Breakthroughs in large-scale computational simulation
    and design} {\tiny (NASA/TM 2002-211747)}

    Unstructured-grid analysis and design across speed range:\\
    Incompressible, compressible, hyper sonic reacting gas

    \begin{tabular}{rl}
      FUN3D & unstructured-grid, incompressible/compressible \\
      LAURA & structured-grid, external hypersonics \\
      VULCAN & structured-grid, internal hypersonics 
    \end{tabular}
  \end{slide}
 
 \begin{slide}
  \title{Why?}
  \begin{itemiz}
   \item Multidisciplinary problems require multiple discipline
   experts, a large infrastructure, and standard interfaces
   \vspace{6pt}
   \item Reduce time from concept to application for vehicles and algorithms
   \vspace{6pt}
   \item Mobility to respond to unforeseen challenges and increase
   software lifespan
  \end{itemiz}

  \title{Research capabilities in a ``production'' code}
  \begin{itemiz}
    \item Infrastructure to evaluate algorithms on large problems
    \item Flexibility for implementing research algorithms 
    \item Stability to suit time-sensitive application needs and to
    release to outside customers
    \item Avoid being encumbered by high-ceremony software development process
  \end{itemiz}
 \end{slide}
 
 \begin{slide}
  \title{Software Versioning System (Control)}
  \begin{itemiz}
    \item Often overlooked or under emphasized
    \item Zeroth principle of software engineering
    \item Learning to work with it and not against it is key to team
    programming (glue)
    \item Safety net
    \item Large impact on ``Truck Number''
    \item Convenient for accounts on multiple machines
    \item Required for automated testing
    \item Not just for software anymore
      \begin{itemiz} 
      \item homework, presentations, configuration files, home accounts
      \end{itemiz}
    \item CVS - \url{https://www.cvshome.org/}
    \item Subversion - \url{http://subversion.tigris.org/}
  \end{itemiz}
 \end{slide}
 
 \begin{slide}
  \title{Software development practices}
   \setlength{\unitlength}{0.1in}%
   \begin{picture}(10,10)(-5,10)
     \put(0,0){\line(3,5){5}}
     \put(0,0){\line( 1, 0){10}}
     \put(10,0){\line(-3, 5){5}}
     \put(0,9){\makebox(0,0)[lb]{Ad hoc}}
     \put(9,-3.4){\makebox(0,0)[lb]{Agile}}
     \put(-10,-3){\makebox(0,0)[lb]{Plan-driven}}
     \put(-3,-6){\makebox(0,0)[lb]{\tiny ``Kleb Triangle''}}
   \end{picture}
  \begin{itemiz}
  \item Ad hoc
    \begin{itemiz} 
    \item ``Code and Fix''
    \end{itemiz}
  \item Plan-driven
    \begin{itemiz} 
    \item Predictive, ``Big up front design''
    \item Delivering to the original contract
    \item Capability Maturity Model (CMM), CMMI 
    \end{itemiz}
  \item Agile
    \begin{itemiz} 
    \item Adaptive, ``Evolutionary design''
    \item Recognizes software development an empirical process that
      can not always be defined
    \item Extreme Programming
    \end{itemiz}
  \end{itemiz}
 \end{slide}

 \begin{slide}
  \title{The Agile Manifesto \normalfont\normalsize values}
  \setlength{\topsep}{0pt}\setlength{\parskip}{5pt}
   \begin{itemiz}
   \item \normalsize individuals and interactions
   \item \small over processes and tools
   \item \normalsize working software
   \item \small over comprehensive documentation
   \item \normalsize responding to change
   \item \small over following a plan
   \item \normalsize customer collaboration
   \item \small over contract negotiation
  \end{itemiz}
   \normalsize 
 \end{slide}
 
 \begin{slide}
  \title{Extreme Programming \normalfont\normalsize values}
  \setlength{\topsep}{0pt}\setlength{\parskip}{5pt}
  \begin{itemiz}
   \item communication
   \item simplicity
   \item feedback
   \item courage
  \end{itemiz}
 \end{slide}
 
 \begin{slide}
  \title{Extreme Programming Practices}
  \begin{itemiz}
    \item[\textit{Sustainable pace}]
      {\small productivity does not increase with hours worked.}
    \item[\textit{Metaphor}]
      {\small guide all development with a simple shared story of how
      the whole system works.}
    \item[\textit{Coding standard}]
      {\small\ write all code in accordance with rules emphasizing
      communication through the code.}
    \item[\textit{Collective ownership}]
      {\small anyone can change any code anywhere in the system at any time.}
    \item[\textit{Continuous integration}]
      {\small integrate and build the system many times a day.}
    \item[\textcolor{mediumGray}{\textit{Small releases}}]
      {\small release new versions on a very short cycle.}
   \end{itemiz}
 \end{slide}
  
 \begin{slide}
  \title{Extreme Programming Practices \small(concluded)}
  \begin{itemiz}
    \item[\textcolor{mediumGray}{\textit{Test-driven development}}]
      {\small any program feature without an automated test simply does
	not exist.}
    \item[\textcolor{mediumGray}{\textit{Refactoring}}]
      {\small restructure the system without
	changing its behavior.}
    \item[\textcolor{mediumGray}{\textit{Simple design}}]
      {\small system should be designed as simply as possible at any
	given moment.}
    \item[\textcolor{mediumGray}{\textit{Pair programming}}]
      {\small two programmers work together at one computer on the same
	task.}
    \item[\textcolor{mediumGray}{\textit{On-site customer}}]
      {\small include a real, live user on the team.}  
    \item[\textcolor{lightGray}{\textit{Planning game}}]
      {\small combine business priorities and technical estimates to
	determine scope of next release.}
  \end{itemiz}
 \end{slide}

 \begin{slide}
  \title{FUN3D Development}
  \begin{itemiz}
    \item[\textit{Sustainable pace}]
      {\small work $\sim$40 hour weeks.}
    \item[\textit{Metaphor}]
      {\small engineering and scientific vocabulary ($\rho$, $u$, $v$, $w$).}
    \item[\textit{Coding standard}]
      {\small\ published to aid portability, automated parsing,
               and collective ownership.}
    \item[\textit{Collective ownership}]
      {\small routinely fix minor bugs or extend methods created 
	by other people. Anyone is allowed to modify any file at 
	anytime through CVS.}
    \item[\textit{Continuous integration}]
      {\small very slow: Linux builds every 2-3 hours, 
	SGI builds every 8-9 hours.}
    \item[\textcolor{mediumGray}{\textit{Small releases}}]
      {\small application members of the team use (CVS), 
              formally 2-3 times a year}
   \end{itemiz}
 \end{slide}
  
 \begin{slide}
  \title{FUN3D Development \small(concluded)}
  \begin{itemiz}
    \item[\textcolor{mediumGray}{\textit{Test-driven development}}]
      {\small limited use in flow solver, 
	extensive use in scripting and grid adaptation.}
    \item[\textcolor{mediumGray}{\textit{Refactoring}}]
      {\small done only when necessary, extremely difficult, 
	painful, and nerve-racking without unit tests.}
    \item[\textcolor{mediumGray}{\textit{Simple design}}]
      {\small born as a result of refactoring and pair programming.}
    \item[\textcolor{mediumGray}{\textit{Pair programming}}]
      {\small limited to mostly debugging, knowledge transfer; 
	impeded by scheduling conflicts. }
    \item[\textcolor{mediumGray}{\textit{On-site customer}}]
      {\small research: we are our own customers?}  
    \item[\textcolor{lightGray}{\textit{Planning game}}]
      {\small comes more naturally in pair programming scheduling. }
  \end{itemiz}
 \end{slide}

 \begin{slide}
  \title{Communication}
  \begin{itemiz}
  \item Collocation
  \item Email list
  \item WikiWikiWeb --  \url{http://c2.com/cgi/wiki?WelcomeVisitors}
  \end{itemiz}
  \title{Scrum status meetings}
  \begin{itemiz}
  \item What they {\bf did} since last meeting
  \item What they will {\bf do} by next meeting
  \item What got {\bf in the way} (impediments)
  \end{itemiz}
  \begin{itemiz}
    \item Quick and efficient meeting style
    \item Reduces the worst management sin (wasting people's time)
    \item The impediments is the often the hardest to express, but the
    most important.
  \end{itemiz}
 \end{slide}
 
 \begin{slide}
   \title{Software Testing} \\
   \emph{All of these can (and should) be automated!}
   \begin{description}
   \item[Programmer's] I want a function that adds vectors, does 
   $f([1, 2], [3, 4])$ return $[4, 6]$? (unit tests)
   \item[Integration] Does my whole system compile and work together?
   \item[Regression] My code gave answer $x$ yesterday, does it give
   answer $x$ today?
   \item[Verification] My code is supposed to be second-order accurate
   in space. What happens when I change the element size?
   \item[Validation] Does my code give the same answer as a wind
   tunnel or fight test measurement?
   \end{description}
 \end{slide}
 
 \begin{slide}
   \title{Unit Testing Frameworks}
   \begin{itemiz}
   \item Goals
     \begin{itemiz}
     \item Interface must allow for the easy creation and management of tests
     \item Minimal additional effort over writing the actual code
     (benefit--cost)
     \item Enable programmers to experience the benefits of test-first
       programming as soon as possible
     \item Legible as documentation
     \end{itemiz}
   \item Flavors
     \begin{itemiz}
     \item \url{http://c2.com/cgi/wiki?TestingFramework}
     \item Full featured (scripting languages)
     \item Minimalistic (four lines of code)
     \item Wrap code to utilize scripting language framework
     \item ``Roll your own''
     \end{itemiz}
   \end{itemiz}
 \end{slide}
 
 \begin{slide}
   \title{ Unit Testing and Test First Programming}
   \begin{itemiz}
   \item Seems trivial at first
   \item Hard to imagine benefit until the first major refactoring or
     code simplification is experienced
   \item Gains power as the number of tests and their coverage increases
   \item Your own custom debugger
   \item Provides a clear completion to an implementation task
   \item Code with a failing test is much easier to fix or extend
   \item Inventing the tests required is generally harder
   \item Code that is easy to test is often simpler and easier to
   read, understand, and extend
   \item Creating tests brings the design to the forefront; design is
   difficult, but it is easiest in small increments
   \end{itemiz}
 \end{slide}
   
 \begin{slide}
   \title{ Discretization error is a major problem }
   \begin{itemiz}
   \item AIAA Drag Prediction Workshop
     \begin{itemiz}
     \item A large number of people applied a large number of codes to
     a single transport configuration
     \item Large spread in results (largest may be programming errors)
     \item Grid converged answers where not demonstrated even for large
       grids (asymptotic range)
     \item Discretization error explicitly identified in reports 
     \end{itemiz}
   \item Multi element high lift and sonic boom calculations
   \item Preventing the characterization of modeling errors (turbulence models)
   \item It is often combated by specifying local grid resolutions by
   hand (requires expert with past experience with similar problems)
   \item Local error estimates have been useful for adaptation but
     often fail for problems that are strongly nonlinear or when
     transported error overcomes the solution
   \end{itemiz}
 \end{slide}
   
 \begin{slide}
   \title{ Adjoint solution }
   \begin{itemiz}
   \item Linearized flow residual and output quantity at a flow state
   \item Efficient method for computing derivatives and design
   \item Existing NASA Langley technology for 3D turbulent design problems
   \item Shows the linearized impact of a equation source term (error)
     on an output function
   \item Indicates the global impact of local errors when combined with
   local error estimates
   \end{itemiz}
 \end{slide}

 \begin{slide}
   \title{Error Estimation and Adaptation}
   \begin{itemiz}
   \item Improve function calculation by predicting and correcting error
   \item Combining flow and adjoint problems
   \item Adapt discretization to improve (not reduce) correction
     \begin{itemiz}
     \item Error estimate can be computed to high accuracy, but it is
       a linear correction
     \end{itemiz}
   \item Based on the 2-dimensional (2D) work of 
     \begin{itemiz}
     \item Venditti and Darmofal (MIT)
     \item M\"uller and Giles
     \end{itemiz}
   \item Intended to avoid manually specifying grid resolution
     to enable design
   \item Requested metric is a Mach Hessian scaled by the adjoint
   error estimate uncertainty
   \end{itemiz}
 \end{slide}

 \begin{slide}
   \title{ Software development Practices }
   \begin{itemiz}
   \item Software version control
   \item Agile software development practices
   \item Types of software testing
   \end{itemiz}
   \title{ Discretization Error }
   \begin{itemiz}
   \item Major issue
   \item Estimation and control by adaptation
   \end{itemiz}
   \title{ Adaptation Mechanics }
   \begin{itemiz}
   \item Limiting factor on applying output based adaptation to design
   and turbulent analysis
   \end{itemiz}
 \end{slide}
 
\end{document}
