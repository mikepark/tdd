\documentclass[twocolumn]{article}

\usepackage{url}
\usepackage{array}

\newenvironment{itemiz}%
  {\begin{list}{}{\raggedright
      \setlength{\itemsep}{2pt}%
      \setlength{\parskip}{4pt}\setlength{\parsep}{2pt}}}%
  {\end{list}}%

\title{Ongoing Research Into Numerical Simulation 
       of Fluid Flows Utilizing Software Developement Practices}
\author{
FUN3D Software Development Team\\
NASA, Hampton, Virginia\\
MikePark@MIT.Edu Mike.Park@NASA.Gov \\
}
\date{24 September 2004}

\setcounter{secnumdepth}{-2}

\begin{document}
  
  \maketitle

  \section{How is software complexity managed when the required 
    infrastructure is increasing?}  Entire Computational Fluid
  Dynamics (CFD) packages have been written by individuals or small
  teams of researchers. They were developed in an ad hoc manner that
  was successful for the size of problem that was attempted. CFD
  packages are continuing to evolve into more and more complex systems
  to handle more classes of problems. They require larger teams to
  assemble and maintain. One way to address this complexity is with
  modern programming practices.

  \begin{itemize}
  \item Research capabilities in a ``production'' code
  \item Software versioning system
  \item Software development practices (Agile)
  \item Extreme Programming principles \textbf{communication simplicity feedback courage}
  \item Extreme Programming \emph{interconnected} practices
    \begin{itemiz}
    \item Sustainable pace
    \item Metaphor
    \item Coding standard
    \item Collective ownership
    \item Continuous integration
    \item Small releases
    \item Test-driven development
    \item Refactoring
    \item Simple design
    \item Pair programming
    \item On-site customer
    \item Planning game
    \end{itemiz}
  \item Software testing
    \begin{itemiz}
    \item Programmer's (unit testing)
    \item Integration
    \item Regression
    \item Verification
    \item Validation
    \end{itemiz}
  \item Unit testing
  \item Test-first programming
  \item Communication -- Scrum status meetings
    \begin{itemiz}
    \item What they {\bf did} since last meeting
    \item What they will {\bf do} by next meeting
    \item What got {\bf in the way} (impediments)
    \end{itemiz}
  \end{itemize}
 
  \section{How is discretization error impacting the solution?}
  Local error estimates for the discretization error have been used to
  describe where increased grid resolution is required to improve a
  solution. These methods have missed the connection between the
  impact of local errors on global output quantities and how these
  local errors are transported. The \emph{adjoint} solution provides
  the critical connection between local errors and global outputs as
  well as how errors are transported.

  \begin{itemize}
  \item Discretiztion error is a major problem
  \item Adjoint solution
  \item Error estimation
  \item Work of Venditti and Darmofal
  \item Sonic boom propagation
  \item Turbulent transport configuration
  \end{itemize}
 
  \section{How is a grid modified to match a desired resolution?}
  Mechanics are required to modify the grid to match a specified grid
  resolution. These mechanics must be able to generate grids with
  anisotropic resolutions for high Reynolds number flows both near
  bodies and in structures like wakes and shocks. These mechanics
  should work seamlessly with the flow solver and error control
  infrastructure.

  \begin{itemize}
  \item Limiting factor in applying the output-based adaptation scheme
  \item The goal is to never need to look at the grid
  \item Work seamlessly with flow solver and error control: must be
  robust, sufficient quality, parallel, etc.
  \item High fidelity surface representation
  \item ONERA M-6 example
  \end{itemize}
 
  \section{Acknowledgments}
  \begin{itemiz}
  \item Dr.~Bil Kleb - \LaTeX~style template and XP definitions 
  \item Dr.~David Venditti and Dr.~David Darmofal - Guidance and 2D results
  \item Beth Lee-Rausch - DLR F-6 adaptation
  \item Entire FUN3D Development Team
  \end{itemiz}

  \section{Resources}

  \subsection{FUN3D}
  \url{http://fun3d.larc.nasa.gov/} \\
  \url{http://hefss.larc.nasa.gov/}

  \subsection{Version Control}
  \url{https://www.cvshome.org/}\\
  \url{http://subversion.tigris.org/}

  \subsection{Portland Pattern Repository's Wiki}
  \url{http://c2.com/cgi/wiki?WelcomeVisitors} \\
  \url{http://c2.com/cgi/wiki?TestingFramework}

  \subsection{Edward Tufte}
  \url{http://www.edwardtufte.com/}\\
  \textit{The Visual Display of Quantitative Information} \\
  and \\
  \textit{The Cognitive Style of Powerpoint}

  \subsection{Andrew Hunt and David Thomas} 
  \url{http://www.pragmaticprogrammer.com/}\\
  \textit{The Pragmatic Programmer: From Journeyman to Master}

  \subsection{ Tom Demarco and Timothy Lister } 
  \textit{Peopleware : Productive Projects and Teams}, 2nd Ed.

\end{document}
